\large
\begin{itemize}
\item The computational cost of the LP is given by:
	  \begin{equation}
	  N_{cost}=\frac{2\cdot N^2}{N-O}+\frac{N^2}{N-O}+P\cdot N
	  \end{equation}
	  This results in $\approx15000$ instructions, using the optimal parameters which was deemed too high to implement in real time. \\
\item The combined LP FXLMS increases the attenuation over a FXLMS system significantly at low $P$s, but at high $P$s the increase is very small. 
	  Therefore to obtain a small $P$ multirate processing can be utilized. \\ 
\item The LP FXLMS frequency response has an increased improvement of attenuation with increased frequency.

%\item The predictor is capable of predicting 14 samples at 48 $k$Hz. When predicting $>$14 samples the predicted signal distorts. If the distorted signal is then passed through the FXLMS algorithm the distortion will not be attenuated.\\
%\item In order to make a real time implementation of the algorithm, certain physical parameters must be considered; firstly the error microphone(3) can not be placed inside the users ear. \\ Secondly the computational cost of the predictor should be decreased.
%Possibly by using a multirate system. %working at a lower sampling frequency as in telecommunications. This could be implemented as a multirate system, to also allow playback of signals with higher fs. 
\end{itemize}

