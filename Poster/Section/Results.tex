

\begin{centering}
	\begin{raggedright}
		\includegraphics[width=2in]{figures/HammingNOP10.pdf}
		Here the parameters framelength (N) and the overlap of the frames (O) are changed in order to determine the values for which the PG is optimized. The optimal coefficients are N=1200 and O=1100, however these can be reduced without hurting the performance significantly.
		\label{fig:HammingNOP10}
	\end{raggedright}


	\begin{raggedleft}
	\includegraphics[width=2in]{figures/DelayRatio.pdf}
	This figure shows the performance of the predictor for different prediction lengths (P). If no system delay is present, e.g. no prediction is needed because the FXLMS algorithm (blue) achieves infinite attenuation. The attenuation of the non predicting algorithm decrease a lot with small delays. The predicting FXLMS algorithm however attenuate speech by 40+ dB for delays up to 14 samples at 48 $k$Hz.
	\label{fig:DelayRatio}
	\end{raggedleft}

	\begin{raggedleft}
	\includegraphics[width=2in]{figures/ComparedConusmerHPOur.pdf}
	The attenuation ot different frequencies for the FXLMS (blue) and the LP FXLMS (red) systems. The frequency content is not changed much by the added predictor, but the attenuation is increased by 30+ dB. 
	\label{fig:OurFreqResp}
	\end{raggedleft}
\end{centering}

%You cannot use floats in the baposter template. However, you can use figure captions by using {\tt \textbackslash captionof} instead of {\tt \textbackslash caption}. This is demonstrated in Fig.~\ref{fig:figlabel}. Moreover, you can also use {\tt \textbackslash label} and {\tt \textbackslash ref} to make references to your figures and/or tables.
%\begin{center}
%	\includegraphics{aau_logo_new}
%	\captionof{figure}{Here is a figure caption}
%	\label{fig:figlabel}
%\end{center}
%As you can see, the text background is not white. If your figures do not have a transparent background, this may look too ugly for you. You can of course change the background colour through the {\tt boxColorOne} option. Alternatively, you can make the background transparent. In Matlab, the following example demonstrates how this is done\par
%{\tt
%	f1 = figure(1);\\
%	set(f1,'Color','none');
%}\par
%You can also use {\tt pgfplots} \cite{pgfplots} for plotting your Matlab data. This is not that hard and the resulting plots are much nicer than Matlab plots, so I will strongly recommend that you have a look at {\tt pgfplots} right here \url{http://sourceforge.net/projects/pgfplots/}.