\documentclass[12pt,a4paper,openright]{article}
\usepackage[utf8]{inputenc}
\usepackage{amsmath}
\usepackage{amsfonts}
\usepackage{amssymb}
\usepackage{graphicx}
\usepackage[left=2.00cm, right=2.00cm, top=2.00cm, bottom=2.00cm]{geometry}
\usepackage{float}
\usepackage{mathtools,xparse}
\newcommand{\Lagr}{\mathcal{L}}
\DeclarePairedDelimiter{\abs}{\lvert}{\rvert}
\DeclarePairedDelimiter{\norm}{\lVert}{\rVert}
\NewDocumentCommand{\normL}{ s O{} m }{%
	\IfBooleanTF{#1}{\norm*{#3}}{\norm[#2]{#3}}_{L_2(\Omega)}%
}
\begin{document}
\title{Poster Presentation}

\section{Introduction}
	ANC -> bad speech -> -> Feedforward -> Delay LP -> We do simulation

\section{Methods}
	FXLMS -> LP (Wiener med Wiener-Hofs fra ACF(LPC))
	
\section{Results of Simulation}
	\begin{itemize}
		\item LP Parameters (Framelength, N - Overlap, O - Prediction Gain, PG)
		\item -> No delay, no LP needed - it just works! -> For delay use LP FXLMS
	-> For optimal parameters gives >40 dB for delays below 14 @ 48 kHz.
		\item Compared in 1/3 -> LP FXLMS gives up to 30 dB
	\end{itemize}
\section{Discussion}
\begin{itemize}
	\item 	High instruction count with optimal parameters too high -> LP FXLMS er bare godt, lower P the better -> Multirate can be used
	\item LP FXLMS compared to FXLMS is better for higher freqs
	\item counter-counter-phase at 2400 Hz, which gives gain instead of attenuation.
\end{itemize}
\section{Conclusion}
	\begin{itemize}
		\item All in all the preposition works -> we attenuate speech greatly - up to 30 dB with LP for P=10
		\item But it is unfeasible this way due to computational cost, with these parameters we need 15000 instructions/sample. Our DSP's are too slow :(
	\end{itemize}

\end{document}


%The highere frequency, the more ("smaller") delays matter.