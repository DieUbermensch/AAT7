\section{Cancellation path}
\subsection{Purpose}
The purpose of this experiment is to determine the impulse response of the cancellation path, i.e. the transfer-function between the "source-loudspeaker" and the "error-microphone (ear)".
		
\subsection{AAU number list}
\begin{table}[h]
	\centering
	\ra{1.3}
	\begin{tabular}{ c c c } \toprule
		{Item}	& {Description} 						& {AAU-no}. \\ \bottomrule 
		1	&	B\&K Head And Torso Simulator "Henry" Type 4128	& 08453		\\
		2	&	B\&K Ear Simulator Type 4159			& 08453		\\
		3	&	Sennheiser HD 200	Headphones			& 33379		\\
		4	&	Roland Edirol UA-25EX Audio Capture		& 64696		\\
		5	&	Vision B3565 Computer					& NaN		\\
		6	&	B\&K Microphone Power Supply Type 2804	& 07304		\\
		%5	&	B\&K \textonehalf'' Condenser Microphone Type 4134 & 61447\\
		%6	&	B\&K Sound Calibrator Type 4231			& 78301		\\ 
		7	&	B\&K Sound Intensity Calibrator Type 3541	& 08597	\\ \bottomrule
	\end{tabular}
	\caption{Table over equipment used in test}
	\label{tab:UsedEquipmentListningCP}
\end{table}

\subsubsection{Diagram}
\begin{figure}[H]
	\centering
	\includegraphics[width=0.7\textwidth]{Shematic_CalcellationPath.pdf}
	\caption{Preliminary Schematic of the Setup.}
	\label{Schematic}
\end{figure}

\subsubsection{Settings/Description}
\label{SettingsCacellationPath}

\begin{itemize}
	\item Item 4 (Sound Capture port) "PHONES" is set to maximum volume
	\item Item 4 (Sound Capture port) "SENS(INPUT 1/L)" is set to + 20 dB		\item Item 5 (Computer) is set to 0 dBFS
\end{itemize}


\subsubsection{Pictures}
\begin{figure}[H]
	\centering
	\includegraphics[width=0.8\textwidth]{CancellationPath_and_Headphones/FullSetup.jpg}
	\caption{Full setup of experiment (Note that external microphone is not in use in the experiment).}
	\label{FullSetupCancellationPath}
\end{figure}

\begin{figure}[H]
	\centering
	\includegraphics[width=0.8\textwidth]{CancellationPath_and_Headphones/CloseUpFace.jpg}
	\caption{Close up of experiment(Note that external microphone is not in use in the experiment).}
	\label{CloseUpCancellationPath}
\end{figure}

\section{Procedure}
\subsection{Set-up}
\begin{enumerate}
	\item Set instruments to comply with the list in \ref{SettingsCacellationPath}
	\item Control, and if necessary, calibrate item 2 using item 7
	\item Place item 3 onto item 1 ("Henry"'s head and ears)
	\item Insert item 2's connector into input of item 6
	\item Connect item 6 to item 4 and connect item 4 to item 5 via USB
	\item Insert item 3's connector into output of item 4
\end{enumerate}
\subsection{Performing the experiment}
\begin{enumerate}
	\item Play "Ref[i].wav"\footnote{[i] indicates the iteration of the experiment} from item 5 through item 4
	\item Record and save, and name "Mic[i].wav" from item 4 onto item 5
	\begin{itemize}
		\item[] Perform this experiment for a total of 5 times
	\end{itemize}
\end{enumerate}

\subsection{Data Extraction}
\begin{enumerate}
	\item Place generated .wav-files in same folder as MATLAB\textsuperscript{\textregistered} script "CancellationPathScript.m".
	\item Run the script "CancellationPathScript.m." in MATLAB\textsuperscript{\textregistered}
\end{enumerate}
This should yield the resulting frequency response, impulse response and finally the transfer function, along other information about the cancellation-path in this set-up.

\subsection{Analysis}
This experiment determined the transfer-function of the cancellation-path.

\subsection{Error Sources}
\begin{itemize}
	\item Placement of the actuators and test-devices
	\item Instrument inaccuracies 
	\item Movement of instruments
	\item Low-frequency background noise
\end{itemize}

\subsection{Conclusion}
We done did it