\section{Measuring a transfer function}\label{Sec:MeasATransFunc}
This section describes the mathematics used in the measurement system. The System can estimate a transfer function H(f) of a system using a reference signal containing all frequencies of interest. Typically  a logarithmic sine sweep. The user should ensure that the entire reference signal is recorded. This probably requires some zero padding of the time signal before playback.
\subsection{Transfer functions in general}
The general way of expressing a transfer function in frequency domain is
\begin{equation}
H(f)=\frac{Y(f)}{X(f)}
\label{Eq:TranferFunction}
\end{equation}
Where $X(f)$ is the Fourier transform of the input signal and $Y(f)$ is the Fourier transform of the the output.  

\subsection{Transfer functions two port system}
In a typical setup you want to meassure the transfer function of a single component, could be a loudspeaker. An amplifier is needed to drive the loudspeaker, but the amplifiers transfer function is not of interest. A two port system allows you to use a meassured signal, from somewhere in the signal chain, as a reference. 
\begin{equation}
H(f)=\frac{Y(f)}{X(f)}=\frac{\mathscr{F}(y(t))\times\mathscr{F}^{\ast}(x(t))}{\mathscr{F}(x(t))\times\mathscr{F}^{\ast}(x(t))}
\label{Eq:TranferFunctionTwoPort}
\end{equation}
Where $X(f)$ is the auto-spectrum of the reference signal and $Y(f)$ cross-spectrum of the reference signal and the measured signal.\\
$x(t)$ and $y(t)$ are of course measured in time domain and to obtain the cross-spectrum and auto-spectrum, $X(f)$ and $Y(f)$ are calculated.
%\begin{equation}
%Y(f)=\mathscr{F}(y(t))\times\mathscr{F}^{\ast}(x(t)) \text{ \& } X(f)=\mathscr{F}(x(t))\times\mathscr{F}^{\ast}(x(t))
%\label{Eq:cross-spectrum}
%\end{equation}
Where $\times$ denotes bin-wise multiplication and $^{\ast}$ denotes the complex conjugate. \\
From equation \ref{Eq:TranferFunction} %and \ref{Eq:cross-spectrum} 
you can find $H(f)$. Taking the inverse Fourier transform of that will give the impulse response of the system. 
\begin{equation}
H(f) = \frac{Y(f)}{X(f)} \xrightarrow{\mathscr{F}^{-1}} h(t)
\label{Eq:Impulseresponse}
\end{equation} 

\subsection{Cross-correlation method}
The mathematics described above can be converted to time domain, and yield the same result. The choice of method is a matter of convenience. If the Fourier transform of the signals are all ready known the above method is easily calculated. On the other hand if the cross-correlation between $y(t)$ and $x(t)$ is already known the method below might be more efficient
\begin{equation}
H(f)=\dfrac{\mathscr{F}(y(t)\star x(t))} {\mathscr{F}(x(t)\star x(t))}=
\dfrac{\mathscr{F}(y(t)\ast x(-t))} {\mathscr{F}(x(t)\ast x(-t))}
\label{Eq:Xcorr method}
\end{equation}  
where $\star $ denotes the cross-correlation and $\ast $ denotes the convolution. %The deconvolution should be made as a division in frequency domain as this is more robust towards noise in the system. 

\subsection{Ref}
%\cite{TutorialMeasurementPowerSpectra}
