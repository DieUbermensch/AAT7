
\section{Prediction in Simulink}


\subsection{Description}
This is a simulation showing the performance of the proposed prediction algorithm. It is not fine tuned in this version but allows for choice of settings corresponding to the need of a specific use case.  


\subsection{Diagram}
\begin{figure} [h]
	\centering
	\includegraphics[width=\textwidth]{../Journal/Code/SimulinkPrediction}
	\caption{Overview of simulation of the prediction algorithm.}
	\label{Fig:PredictionSimulink}
\end{figure}


\subsection{Implementation}

\subsubsection{variables}
A lot of variables, are declared in the simulation and they are not all described here. The ones described are only the tunable parameters, that should be set in the prediction problem that should be solved. 
\begin{lstlisting} [language=MATLAB]
N; 				%is the frame length used as memory in the prediction and the prediction order
Overlp;			%is the overlap in the buffer
prediction; 	%is how far the algorithm should predict 
fs; 			%is the sample frequency of the input signal 
w=hamming(N)	% could be replaced by other window functions.
\end{lstlisting}
The following sections will show the code for the individual blocks. \\
The ACF estimation estimates the autocorrelation function using the equation...
\subsubsection{ACF Estimation}
\begin{lstlisting} [language=MATLAB]
function r = fcn(x,w,updateRate,N)
persistent temp;

if isempty(temp)
	temp=zeros(N);
end

for lag = 0:N
	for n = lag+1:N
		temp(lag+1,1+N-n) = (x(n)*w(1+N-n)*x(n-lag)*w(1+N-n+lag));
	end
end
r=sum(temp')';
end
\end{lstlisting}

\subsubsection{Lenvinson Durbin}
\begin{lstlisting} [language=MATLAB]
function a = fcn(r,N)
a=levinson(r,N)';
end
\end{lstlisting}

\subsubsection{Recursive filter}
\begin{lstlisting} [language=MATLAB]
function xhat = fcn(x,a,prediction,N)
%#codegen
persistent xIn;     %buffer for input

if isempty(xIn)
	xIn=zeros(1,N+prediction);
end

% throw the oldest x away and keep indsert the newest one
xIn = [xIn(2:N) x zeros(1,prediction)]; 

for j = 1:prediction % predict "prediction" samples ahead. 
	xIn(1,N+j) = -a(2:end,1)'*fliplr(xIn(1,1+j:N-1+j))';
end
xhat = xIn(1,N+prediction);
end
\end{lstlisting}

\subsection{Prediction Gain}
Prediction gain PG is used to evaluate the performance of the predictor. PGis given as:
\begin{equation}
	PG=10 \cdot log_{10}(\frac{\sigma^2_{input_signal}}{\sigma^2_{error}})
\end{equation} 
A PG=0 means no prediction can be achieved. A higher PG means better prediction.

\subsection{Test setup.}
It is required to run the simulation with a lot of different parameters. Rather than going for a random trial and error approach a systematic variation approach was chosen. If a person should control the simulations manually a lot of time would be spent, waiting for the algorithm to run. Therefore an automated test setup was constructed. The automation consists of 4 parts. 
\begin{enumerate}
	\item The prediction simulation
	\item An initiation 
	\item Saving data
	\item A script controlling parameters. 
\end{enumerate}  

Below an example of the 3 controlling scripts are shown. The script saving data is the only one that is constant. The other scripts should be modified to calculate using the desired combination of parameters.

\textbf{Init.m}
\begin{lstlisting} [language=MATLAB]
fs = 48000;
ts = 1/fs; 
% Predictor 
% N = 50; %Frame
% M = N;   %Coefficients
% overlap = 0; %updaterate
% prediction = 1; % How far should the prediction be
w = hamming(N); 
% w = rectwin(N);
%w = barnwell(N,0.98);
% w = fliplr(w);
\end{lstlisting}

\textbf{SaveData.m}
\begin{lstlisting} [language=MATLAB]
	PG=10*log10(var(Ref)/var(ErrorSig));
	
	fname = sprintf(['fs' num2str(fs ) '_N' num2str(N) '_O' num2str(overlap) ...              '_M' num2str(M) '_P' num2str((prediction)) '.mat']);
	
	save(fname,'fs','N','overlap','M','prediction','PG');
\end{lstlisting}

\textbf{RunSim.m}
\begin{lstlisting} [language=MATLAB]
	clear 
	prediction=10;
	
	for N=100:100:3000
		M=N;
		for overlap=0:100:N-100
			sim('Predictora.slx')
			disp(['overlap=' num2str(overlap) 'done'])
		end
		disp(['N=' num2str(N) ' done'])
	end
	
	disp('I am done')	
\end{lstlisting}


Afterwards the script "loadData.m" can be used to structure the stores .mat  files in a single array.\\
\textbf{LoadData.m}
\begin{lstlisting} [language=MATLAB]
	clear
	close all
	fs=48000;
	prediction=10;
	counter=0;
	
	for N=100:100:3000
	M=N;
		for overlap=0:100:N-100
			fname = sprintf(['fs' num2str(fs) '_N' num2str(N) '_O' num2str(overlap) '_M' num2str(M) '_P' num2str((prediction)) '.mat']);
			load(fname)
			counter=counter+1;
			result(counter,:)=[fs N overlap M prediction PG]; 
		end
	end
	
	disp('I am done')
\end{lstlisting}

 
\subsection{Determining N and O}
Firstly the performance with different prediction orders is measured. This is the frame length parameter (N). \\ 
Here the "prediction" length is set to 10. Overlap is set varied from 0 to N-100. The used window is the Hamming window. 



