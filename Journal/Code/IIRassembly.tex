
\section{IIR filter in assembly}

\subsection{Description}
This function filters an input signal with a set of filter coefficients with a given length (order).

\subsection{Syntax} 
\begin{lstlisting}
	extern Int16 IIR(Int16 *x, Int16 *b, Int16 order, Int16 index);
\end{lstlisting}

\subsection{Diagram}
\begin{figure} [h]
	\centering
	\includegraphics[width=0.6\textwidth]{../Journal/Code/IIRfilter}
	\caption{Diagram of a 2nd order Direct-form II IIR filter}
	\label{Fig:FIR_filter}
\end{figure}


\subsection{Implementation}

\subsubsection{variables}
\begin{lstlisting}
	Int16 	*x 		//The input signal
	Int16	*b		//The filter coefficients
	Int16	order	//The order of the filter (length of x & b)
	Int16	index	//The index of the newest x sample
	Int16	return	//Returns the output of the filter (y(n))
\end{lstlisting}

\subsubsection{code}
\begin{lstlisting} [language={[x86masm]Assembler}]
	.global _FIR;

;------------------------------------------------------
;		void FIR(  	Int16 *x,	=> AR0
;			  		Int16 *b, 	=> AR1
;			   		Int16 N, 	=> T0
;			   		Int16 i)		=> T1



_FIR:	
MOV #0, AC0					; Clear AC1
MOV mmap(AR0),BSA01			; Set base Adress for buffer
MOV mmap(T0), BK03			; Set buffer size to filter order
MOV *(#T1),AR0				; Start from ith sample
bset AR0LC					; Set pointer as circular
SUB #1,T0					; Setup Repititions
MOV T0,CSR					; Setup Repititions
RPT CSR						; Repeat order times
MAC *AR0-,*AR1+,AC0 		; Do filter convolution
SFTL AC0, #-15				; Bit shift to fit Q15
MOV AC0,T0					; Output sample
RET
\end{lstlisting}