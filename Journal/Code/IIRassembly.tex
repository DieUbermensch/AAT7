
\section{IIR filter in assembly}

\subsection{Description}
This function filters an input signal with a set of filter coefficients with a given length (order).

\subsection{Syntax} 
\begin{lstlisting}
	Int16 simpleiir(Int16 *buffer,Int16 iirorder,Int16 * output, Int16 buffsize,Int16 * acoeffs, Int16 * bcoeffs);
\end{lstlisting}

\subsection{Diagram}
\begin{figure} [h]
	\centering
	\includegraphics[width=0.6\textwidth]{../Journal/Code/IIRfilter}
	\caption{Diagram of a 2nd order Direct-form I IIR filter}
	\label{Fig:FIR_filter}
\end{figure}


\subsection{Implementation}

\subsubsection{variables}
\begin{lstlisting}
	Int16 	*buffer 	//input buffer
	Int16	*iirorder	//Number of biquads
	Int16	*output	//output buffer
	Int16 	buffsize	//size of the buffer
	Int16 * acoeffs		//a coefficients
	Int16 * bcoeffs	//b coefficients
\end{lstlisting}

\subsubsection{code}
\begin{lstlisting} [language={[x86masm]Assembler}]
	.global _simpleiir
       
_simpleiir:
      pshm  ST1_55             ; Save ST1, ST2, and ST3
      pshm  ST2_55
      pshm  ST3_55
      
      mov #0,AC0				;clear accumulator
      add #1,AR1				;go to y(n-1) 
      sub #1,T0                 ;Number of samples - 1
      mov T0,BRC0            	;Setup loop to blkSize-1
      rptblocal sample_loop-1
      mac *AR0+,*AR3+,AC0 		;x(n) 	to accumulator
      mac *AR0+,*AR3+,AC0 		;x(n-1) to accumulator
      mac *AR0+,*AR3+,AC0 		;x(n-2) to accumulator
      mac *AR1+,*AR2+,AC0 		;this should be y(n-1)    
      mac *AR1+,*AR2+,AC0 		;this should be y(n-2)   
      mov  AC0,*AR0 
sample_loop
      mov  HI(AC0),T0
                   
      popm  ST3_55                 ; Restore ST1, ST2, and ST3
      popm  ST2_55
      popm  ST1_55
      ret
	
      .end
\end{lstlisting}

\subsection{Test}

The test use the exact same set-up and procedure as appendix J. 

Here, a low pass second order filter have been designed using FDA Tool in Matlab. The following coefficients have been generated: a (line 1) and b (line 2) with an overall gain of 0.2180 (around 13dB)
\begin{lstlisting}
-0,351328328	 0,329689257	
 1				 1,999229009	  1
\end{lstlisting}


\begin{figure}[H]
	\centering
	\tikzsetnextfilename{FrequencyResponseDesign}
	\input{../Journal/Code/Test/FrequencyResponseDesign.tex}
	\caption{Attenuation achieved by the system for different system delays.}
	\label{Fig:FrequencyResponseDesignIIR}
\end{figure}

\begin{figure}[H]
	\centering
	\tikzsetnextfilename{FrequencyResponse}
	\input{../Journal/Code/Test/FrequencyResponse.tex}
	\caption{Attenuation achieved by the system for different system delays.}
	\label{Fig:FrequencyResponseIIR}
\end{figure}
