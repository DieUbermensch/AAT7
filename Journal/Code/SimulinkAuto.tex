\section{Automation of tests in Simulink} \label{sec:SimulinkAuto}

All files from this appendix can be found in folder: \\
\path{Attachment://Appendix R - AutomationOfTestInSimulink}

\subsection{Description}
This appendix describes an automated test set-up used with simulations, to run the same simulation multiple times with different parameters. 
\subsection{Variables}
Numerous variables are declared in the simulations and they are not all described here. The ones described are only the tunable parameters, that should be set in the prediction problem that should be solved. 
\begin{lstlisting} [language=MATLAB, caption=Variables.]
N; 				%is the frame length used as memory in the prediction and the prediction order
overlap;		%is the overlap in the buffer "O"
prediction; 	%is how far the algorithm should predict "P"
fs; 			%is the sample frequency of the input signal 
w=hamming(N)	% could be replaced by other window functions
M;				% is equal to N in all cases
u 				% in case a FXLMS algorithm is included in the test
\end{lstlisting}

\subsection{Functions}
 The automation consists of four parts. 
\begin{enumerate}
	\item The Simulink set-up under test. 
	\item An initiation. 
	\item Saving data.
	\item A script controlling parameters. 
\end{enumerate}  

Below an example of the three controlling scripts are shown. The  scripts should be modified to run the simulator using the desired combination of parameters.

\textbf{Init.m}\\
If a variable is not controlled in "RunSim.m" it should be set here. 
\begin{lstlisting} [language=MATLAB, caption=init.m.]
fs = 48000;
ts = 1/fs; 
% Predictor 
% N = 50; %Frame
% M = N;   %Coefficients
% overlap = 0; %updaterate
% prediction = 1; % How far should the prediction be
w = hamming(N); 
% w = rectwin(N);
%w = barnwell(N,0.98);
% w = fliplr(w);
\end{lstlisting}

\textbf{SaveData.m}\\
It is possible to modify the save function to store all variables, including the calculated audio. This however results in approximately 5 $M$B of data pr. simulation of 5 seconds.  
\begin{lstlisting} [language=MATLAB, caption=SaveData.m.]
	PG=10*log10(var(Ref)/var(ErrorSig));
	
	fname = sprintf(['fs' num2str(fs ) '_N' num2str(N) '_O' num2str(overlap) ...              '_M' num2str(M) '_P' num2str((prediction)) '.mat']);
	
	save(fname,'fs','N','overlap','M','prediction','PG');
\end{lstlisting}

\textbf{RunSim.m}
\begin{lstlisting} [language=MATLAB, caption=RunSim.m.]
	clear 
	prediction=10;
	
	for N=100:100:3000
		M=N;
		for overlap=0:100:N-100
			sim('Predictora.slx')
			disp(['overlap=' num2str(overlap) 'done'])
		end
		disp(['N=' num2str(N) ' done'])
	end
	
	disp('I am done')	
\end{lstlisting}

\subsection{Data Extraction}
After running the automated test, the script "LoadData.m" can be used to structure the saved .mat files into a single matrix.\\
\textbf{LoadData.m}
\begin{lstlisting} [language=MATLAB,caption=LoadData.m.]
	clear
	close all
	fs=48000;
	prediction=10;
	counter=0;
	
	for N=100:100:3000
	M=N;
		for overlap=0:100:N-100
			fname = sprintf(['fs' num2str(fs) '_N' num2str(N) '_O' num2str(overlap) '_M' num2str(M) '_P' num2str((prediction)) '.mat']);
			load(fname)
			counter=counter+1;
			result(counter,:)=[fs N overlap M prediction PG]; 
		end
	end
	
	disp('I am done')
\end{lstlisting}

\subsection{Test}
An example of the output from an automated test can be seen on \autoref{fig:AutomationFiles}.

\begin{figure} [H]
	\centering
	\includegraphics[width=0.5\textwidth]{Automation/SavedFilesSim}
	\caption{A file store by the automated test viewed in Matlabs file preview.}
	\label{fig:AutomationFiles}
\end{figure}


\subsection{Conclusion}
A working framework for automating tests in Simulink has been created. 

