%pagestyle{fancy} %enable headers and footers again

\begin{comment}
\pdfbookmark[0]{Danish title page}{label:titlepage_en}
\aautitlepage{%
  \englishprojectinfo{
    Project Title %title
  }{%
    Analoge kredsløb og systemer %theme
  }{%
    P3: 2. September 2014 - 17. December 2014 %project period
  }{%
    14gr313 % project group
  }{%
    %list of group members
    Amalie Vistoft Petersen\\
    Mikkel Krogh Simonsen\\
    Rasmus Gundorff Sæderup\\
    Simon Bjerre Krogh\\
    Thomas Kær Juel Jørgensen\\
    Thomas 'Godlike' Rasmussen
  }{%
    %list of supervisors
    Tom S. Pedersen

  }{%
    9 % number of printed copies
  }{%
    \today % date of completion
  }%
}{%department and address
  \textbf{Institut for Elektroniske Systemer}\\
  Fredrik Bajers Vej 7\\
  DK-9220 Aalborg Ø\\
  }{% the abstract
  Here is the abstract
}

\cleardoublepage

\end{comment}

\selectlanguage{english}
\pdfbookmark[0]{Titelblad}{label:titelblad}
\aautitlepage{%
  \danishprojectinfo{
    Active Noise Control of Speech in Headphones using Linear Prediction%title
  }{%
    Digital Signal Processing \\Applied to Acoustical Signals  %theme
  }{%
    7. Semester %project period
  }{%
    16gr761 % project group
  }{%
    %list of group members
    Christian Claumarch\\
    Kasper Kiis Jensen\\
    Maxime Démurger\\
    Mikkel Krogh Simonsen\\
    Oliver Palmhøj Jokumsen
  }{%
    %list of supervisors
    Flemming Christensen 
    }{    
    }{%
    4 % number of printed copies
  }
  {%
    20. December, 2016%\today % date of completion
  }%
}{%department and address
  \textrm{\textbf{Institute of Electronic Systems}\\
  Fredrik Bajers Vej 7\\
  DK-9220 Aalborg Ø\\}
 }{
Active Noise Control (ANC) is a widely used technique in consumer headphones for attenuating noise. ANC is useful for attenuation of periodic noise e.g. machinery but has limited ability to attenuate quasiperiodic noise e.g. speech (50 Hz -- 4000 Hz). This paper therefore focuses on improving attenuation of speech. Feedforward ANC systems are widely used, where an FIR-filter is adapted by a Filtered-$x$ Least Mean Squares (FXLMS)-algorithm. The main problem when implementing ANC is delays in converters. A tested $\Sigma\Delta$-converter has a delay of 225 $\mu$s -- 900 $\mu$s. A Linear Prediction (LP) method combined with multirate processing is proposed to compensate for the introduced conversion delays. A signal sampled at 192 $k$Hz decimated to 48 $k$Hz requires prediction of 10 samples.   Cascaded Wiener filtering is therefore used to predict 10 samples $\approx$ 225 $\mu$s at 48 $k$Hz. 
}