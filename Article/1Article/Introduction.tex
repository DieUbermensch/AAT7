%% Introduction
\section{Introduction}
\lettrine[lines=2]{A}{ctive} Noise Control (ANC) is becoming a widely used technique to attenuate noise in consumer headphones. One of the limiting factors in ANC is the upper frequency which is attenuated, therefore it is of interest to examine why performance decreases with increased frequency and how to extend bandwidth. The bandwidth is determined by the system delay which is introduced by sampling and processing the signal.   

The general solutions in ANC are described thoroughly by Hansen et al. \cite{Hansen2}, \cite{Hansen}. The solution used as a reference in this paper is the digital feedforward system using the Filtered-X Least Mean Squares (FXLMS) method. To increase bandwidth a prediction algorithm is proposed as a potential solution. In order to predict certain signal characteristics must be known, here speech is chosen as the noise source. Prediction of speech is described by Wai C. Chu \cite{Speech}. In this paper we will combine the reference solution with the prediction of speech in an attempt to increase the bandwidth of a real time implementation.  

The paper is split into three parts. The first part describes the method used in the reference solution and how to predict speech using Linear Prediction (LP). The second part describes the simulated performance of the reference and the combined solution. The third part describes the real time implementation of the LP algorithm and verifies the increase in performance of the LP ANC system.  
        




%  The scope of this paper is not to derive a new ANC algorithm, but rather to expand the existing FXLMS algorithm by prediction. The goal of this modification is to achieve increased performanec, especially at higher frequencies.\\
% The application of the system is cancellation of speech in a call centre. The choice of a specific use case allows the frequency range and signal type of interest to be defined before designing the system. Call centres is an especially interesting environment for an ANC system, because the unwanted noise and the wanted signal have the same  characteristics as they are both speech. \\
% The paper is split into three parts. The first part examines the demands for an ANC system to be used in a call centre. The second part discus the algorithm used and shows preliminary results from simulations. The third part describes the real time implementation of the algorithm and verifies the performance of the ANC system.  




% \lettrine[lines=2]{A}{ctive} Noise Control (ANC) is a field of study, where a lot of algorithms are already known. The scope of this paper is not to derive a new ANC algorithm, but rather to expand the existing FXLMS algorithm by prediction. The goal of this modification is to achieve increased performanec, especially at higher frequencies.\\
% The application of the system is cancellation of speech in a call centre. The choice of a specific use case allows the frequency range and signal type of interest to be defined before designing the system. Call centres is an especially interesting environment for an ANC system, because the unwanted noise and the wanted signal have the same  characteristics as they are both speech. \\
% The paper is split into three parts. The first part examines the demands for an ANC system to be used in a call centre. The second part discus the algorithm used and shows preliminary results from simulations. The third part describes the real time implementation of the algorithm and verifies the performance of the ANC system.  




%\lettrine[lines=2]{A}{ctive} Noise Control (ANC) is a field of study, where a lot of algorithms are already known. \todo[inline]{$MENTION A FEW$} The scope of this paper is not to derive a new ANC algorithm, but rather to use an existing algorithm in a practical application. The application is cancellation of speech in a call centre The choice of a specific use case allows the frequency range and signal type of interest to be defined before designing the system. Call centres is an especially interesting environment for an ANC system, because the unwanted noise and the wanted signal have the same  characteristics as they are both speech. \\
%The paper is split into three parts. The first part examines the demands for an ANC system to be used in a call centre. The second part discus the algorithm used and shows preliminary results from simulations. The third part describes the real time implementation of the algorithm and verifies the performance of the ANC system.  