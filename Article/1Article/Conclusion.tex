\section{Discussion}
%The system parameters for the predictor are chosen based on an examination of \autoref{tb:fsPredict} for $f_s$, \fxnote{ref to fs:delay} for $P$, \autoref{fig:PredictN} for $N$, \autoref{fig:PredictO} for $O$ and \autoref{fig:PredictM} for M.
It can be seen from \autoref{tb:fsPredict} that a higher $f_s$ yields a higher PG. This can be attributed to prediction being dependent on how much time needs to be predicted, less time needs to be predicted at higher $f_s$ as shown in \autoref{tab:DelayResults}.  
%Which is because if the same number of samples is predicted at different $f_s$ then the higher the $f_s$ the less time needs to be predicted out in the future shown in \fxnote{ref to fs:delay} given that a higher $f_s$ results in a smaller converter delay. 
Therefore a higher $f_s$ is to be prefered up until a DSP cannot handle the computational load or uses to much power. For this simulation $f_s$ is chosen to be $48$ $k$Hz to yield a high PG. 
Using $f_s$ equal to $48$ $k$Hz the required $P$ is 43 samples. With $P=43$ the optimal parameters for the predictor was found in \autoref{fig:Predict43} to be $N=1700$ and $O=1600$ based on maximizing PG. The optimum PG is equal to 5.4 dB. 
\\\\
The converter delay at $f_s=192$ $k$Hz is a quarter of the delay at $f_s=48$ $k$Hz. To utilize the fast conversion time multirate processing could be used. If the signal is sampled at 192 $k$Hz, decimated to 48 $k$Hz, prediction of roughly 10 samples is adequate. The decimation and subsequent interpolation should be done using 1. order IIR filters to avoid delays. In \autoref{fig:Predict10} the optimum $N$ and $O$ are 1200 and 1100 respectively with a PG of 10.0 dB. 
\\\\
The LPC order $M$ can be chosen to be much smaller than $N$ according to \cite{Speech}, however by experiment it was found that this is only true for $P$ small. With $P=10$, $M=N$ is required to yield a high PG. 
\\\\
\autoref{fig:PredictP} shows that PG decreases with higher $P$ however the visualization does not show the subjective sound quality of the predicted signal. At $P$ greater than approximately 14 the signal is distorted.    
\\\\
When the two simulated ANC systems with and without LP are compared as shown in \autoref{Fig:delayRatio} it is obvious that the system with LP is superior to the system without LP. With a system delay of 10 samples an increased attenuation of 34 dB is achieved. At 43 samples system delay an increased attenuation of only 6.2 dB is achieved. As mentioned distortion is introduced for high P's which results in a large decrease in attenuation as shown in \autoref{Fig:delayRatio} because the distortion is not attenuated by the FXLMS.
\\\\
The frequency response of the feedforward LP FXLMS shown in \autoref{fig:ANCcompareALL} is similar to the frequency response of the feedforward FXLMS however with a higher attenuation. Listening to the output of the two ANC systems the difference in attenuation is clearly audiable. Showing that the concept of LP combined with FXLMS yields higher attenuation of speech in simulation     
\\\\
If implementing a feedforward LP FXLMS system in a product certain previous assumptions must be reconsidered. In the test setup shown in \autoref{fig:SystemOverview} the error microphone is placed inside the ear which is of course not practical in a product. This can be solved by the use of different techniques e.g. virtual sensing or virtual sound pressure as described in \cite{Hansen2}. If a real time implementation is to be made a lower $f_s$ should be considered. The predictor is very computational heavy and therefore lowering $f_s$ to 8 $k$Hz should be considered. 8 $k$Hz is often used for speech encoding where Wiener filtering is also used \cite{Speech}.



\section{Conclusion}
The proposed method of combining FXLMS with LP ($P=10$) has proven to improve the attenuation of a speech signal (50-4000 Hz) by 34 dB compared to a FXLMS system without LP. This is because the LP algorithm is capable of compensating for the delay introduced by sampling and reconstruction. The sampling delay of a $\Sigma\Delta$ - converter can be reduced by sampling at a high frequency and processing in multirate also yielding a smaller computational cost. This will be required if implementing the simulated system.    
\\\\
The predictor is reliant on short term stationary signals in order to predict future samples. The method used is Wiener filtering, which is often used in speech encoding, coupled in cascade for predicting $P$ samples. This method of predicting results in a PG of 10 dB when predicting 10 samples and the speech is understandable.    
\\\\
In order to utilize the proposed system in real time certain challenges must be overcome. The primary challenge being the computational cost of implmenting LP. If however these challenges can be overcome the potential increase in attenuation is large.   



\section*{Acknowledgements}
The group would like to thank Flemming Christensen for supervision, Claus Skipper Jensen for technical assistance and Søren Bech for lending a pair of Beoplay H8 for testing. 
