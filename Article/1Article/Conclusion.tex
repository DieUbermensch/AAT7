\section{Discussion}
%The system parameters for the predictor are chosen based on an examination of \autoref{tb:fsPredict} for $f_s$, \fxnote{ref to fs:delay} for $P$, \autoref{fig:PredictN} for $N$, \autoref{fig:PredictO} for $O$ and \autoref{fig:PredictM} for M.
It can be seen from \autoref{tb:fsPredict} that the higher the $f_s$ the higher the PG. This can be attributed to prediction being dependent on how much time needs to be predicted, less time needs to be predicted at higher $f_s$ as shown in \autoref{tab:DelayResults}.  
%Which is because if the same number of samples is predicted at different $f_s$ then the higher the $f_s$ the less time needs to be predicted out in the future shown in \fxnote{ref to fs:delay} given that a higher $f_s$ results in a smaller converter delay. 
Therefore a higher $f_s$ is to be prefered up until a DSP cannot handle the computational load or uses to much power. For this simulation $f_s$ is chosen to be $48$ $k$Hz to yield a higher PG. If a real time implementation is to be made a lower $f_s$ is probably to be considered. A $f_s$ of 8 $k$Hz is often used for speech encoding where Wiener filtering is also used \cite{Speech}.       
\\\\
Using $f_s$ equal to $48$ $k$Hz the required P is 43 samples. With $P=43$ the optimal parameters for the predictor was found using \autoref{fig:Predict43} to be $N=1200$ and $O=1100$ based on maximizing PG. The optimum PG is equal to 10.0 dB. Because PG for $P=43$ is low a simulation of $P=10$ is made to determine if the optimal $N$ and $O$ have changed. From \autoref{fig:Predict10} $N$ and $O$ are verified to have same optimal values as for $P=43$. If computational cost was to be considered when chosing $N$ and $O$ lower values can yield almost similar PG. As an example using $N=600$ and $O=500$ a PG of 9.5 dB is achieved.  





% Using $f_s$ equal to $48$ kHz P can be determined from \fxnote{ref to fs:delay} to be $TBD$ $\mu$s which is equal to $TBD$ samples. $N$ is detemined by examinating \autoref{fig:PredictN}. PG is very low for P equal 40 therefore parameters for P equal 10, 20 and 40 are determined. Here it can be seen that $N$ increases up until approximately $400$ where PG becomes stable. Therefore $N$ is determined to be $400$ because a larger $N$ does not result in a higher PG. The same examination is done for $O$ and $M$ from \autoref{fig:PredictO} and \autoref{fig:PredictM}.



% This results in $O$ determined to $TBD$ and $M$ determined to $TBD$.   


% $f_s$ is determined by examining if a higher $f_s$ will result in a higher PG, shown on \autoref{tb:fsPredict}. This is assumed because $f_s$ is linear proportional to the converter delay as seen on \fxnote{Fs:delay figure in introduction}. This means that an equal amount of samples which must be predicted independent on $f_s$, which results in the higher the $f_s$ the less time needs to be predicted. 

% The determination of the parameters are chosen based on having the largest PG with the lowest value of $N$, $O$ and $M$. $N$ is determined to be 500, $O$ is determined to be 300 and $M$ is determined to be 50. These values will be used in the predictor.




\subsection*{Real time limitations}
Placement of error microphone.



\section{Conclusion}

\section*{Acknowledgements}
