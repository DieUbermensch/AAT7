\section{Discussion} \label{sec:discussion}
%The system parameters for the predictor are chosen based on an examination of \autoref{tb:fsPredict} for $f_s$, \fxnote{ref to fs:delay} for $P$, \autoref{fig:PredictN} for $N$, \autoref{fig:PredictO} for $O$ and \autoref{fig:PredictM} for M.
Table \ref{tb:fsPredict} shows that a higher $f_s$ yields a higher $PG$. This can be attributed to prediction being dependent on the amount of time to be predicted, as less time needs to be predicted at higher $f_s$ as shown in \autoref{tab:DelayResults}.  
%Which is because if the same number of samples is predicted at different $f_s$ then the higher the $f_s$ the less time needs to be predicted out in the future shown in \fxnote{ref to fs:delay} given that a higher $f_s$ results in a smaller converter delay. 
Therefore a higher $f_s$ is preferred up until a DSP cannot handle the computational cost. For this simulation $f_s$ is chosen to be $48$ $k$Hz to yield a high PG. 
Using $f_s$ equal to $48$ $k$Hz the required $P$ is 43 samples.  
\\\\
The converter delay at $f_s=192$ $k$Hz is a quarter of the delay at $f_s=48$ $k$Hz. To utilize the fast conversion time, multirate processing could be used. If the signal is sampled at 192 $k$Hz, then decimated to 48 $k$Hz, prediction of roughly 10 samples is adequate. The decimation and subsequent interpolation should be done using 1st order IIR filters to avoid delays.  
\\\\
According to \cite{Speech}, the LPC order, $M$, can be chosen much smaller than $N$, however by experiment it was found that this is only true for $P$ small. With $P=10$, $M=N$ is required to yield a high $PG$. 
\\\\
Figure \ref{fig:PredictP} shows that $PG$ decreases with higher $P$, however the visualization does not show the subjective sound quality of the predicted signal. As $P$ increase the sound quality drops due to artefacts.    
\\\\
When the two simulated ANC systems with and without LP are compared as shown in \autoref{Fig:delayRatio} it is obvious that the system with LP is superior to the system without LP. With a system delay of 10 samples, an increased attenuation of 34 dB is achieved. At 43 samples system delay, an increased attenuation of only 6.2 dB is achieved. %As mentioned distortion is introduced for high $P$s which results in a large decrease in attenuation as shown in \autoref{Fig:delayRatio} because the distortion is not attenuated by the FXLMS.
\\\\
The frequency response of the feedforward LP FXLMS shown in \autoref{fig:ANCcompareALL} shows that the attenuation is  larger than the feedforward FXLMS at frequencies above 125 Hz. For the feedforward FXLMS algorithm a gain is present around 2400 Hz. This is due to the counterphase signal being 180$^{\circ}$ delayed at this frequency, resulting in positive interference.\\\\
When listening to the output of the two ANC systems the difference in attenuation is clearly audible. This shows that the concept of LP combined with FXLMS yields higher attenuation of speech in simulation.     
\\\\
If a feedforward LP FXLMS system is to be implemented in a product certain assumptions must be reconsidered. In the test setup shown in \autoref{fig:SystemOverview} the error microphone is placed inside the ear which is of course not practical in a product. This can be solved by the use of different techniques e.g. virtual sensing or virtual sound pressure as described in \cite{Hansen2}. The computational cost of the predictor is very high. With the chosen parameters, >50,000 instructions are needed per sample, as can be seen from \autoref
{eq:Cost}. The cost can be lowered by lowering $f_s$ or by choosing smaller $N$ and $O$, however this will decrease the performance of the system.  

% If a real time implementation is to be made, a lower $f_s$ should be considered. The predictor is very computationally heavy and therefore lowering $f_s$ to 8 $k$Hz should be considered. 8 $k$Hz is often used for speech encoding where Wiener filtering is also used \cite{Speech}. 



\section{Conclusion}
The proposed method of combining feedforward FXLMS with LP ($P=10$) has proven to improve the attenuation of a speech signal (50 Hz -- 4000 Hz) by 34 dB compared to a feedforward FXLMS system without LP. This is because the LP algorithm is capable of compensating for the delay introduced by sampling and reconstruction. The sampling delay of a $\Sigma\Delta$-converter can be reduced by sampling at a high $f_s$ and processing in multirate, which would also yield a smaller computational cost. %This will be required if implementing the simulated system.    
\\\\
The predictor is reliant on short term stationary signals in order to predict future samples. The method used is Wiener filtering, which is often used in speech encoding, coupled in cascade for predicting $P$ samples. This method of prediction results in a $PG$ of 10 dB when predicting 10 samples. In order to utilize the proposed system in real time certain challenges must be addressed. The primary challenge is the computational cost of >50,000 instruction per sample if implementing LP.    



\section*{Acknowledgements}
We would like to thank Associate Professor Flemming Christensen for supervision, Engineering Assistant Claus Vestergaard Skipper for technical assistance and Professor Søren Bech for lending a pair of Beoplay H8 for testing. 
