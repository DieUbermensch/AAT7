\section{Speech} \label{sec:Speech}
This section will give a short introduction to speech examine the characteristics of speech. 

"Speech is generated as an acoustic wave that is radiated from the nostrils and the mouth when air is expelled from the lungs with the resulting flow of air petubed by the constrictions insede the body."

Speech can be intepreted as an acoustic filter with time varying frequency response formed by the different parts of the human speech production system. Speech can be classified in to two main classes: voiced or unvoiced, where voiced sounds are characterized by strong periodicity in the sound with the fundamental frequency refered to as the pitch frequency, or simply pitch. Where the pitch ranges from 50 to 250 Hz for men and 120 to 500 Hz for women. Unvoiced sounds on the other hand are random. For a more detailed explanation refer to \fxnote{Source}. 

An example of the "problems" uttered by a man is given in \autoref{fig:voicedUnvoiced}, where the characteristics of voiced and unvoiced can be seen. 

\begin{figure}[H]
	\centering
	\includegraphics[width=0.6\textwidth]{figures/Speech/VoicedvsUnvoiced}
	\caption{Example of speech waveform uttered by a male subject about the word ‘‘problems.’’ The expanded views of a voiced frame and an unvoiced frame are shown, with the magnitude of the Fourier transorm plotted. The frame is 256 samples in length.}
	\label{fig:voicedUnvoiced}
\end{figure}   

From the example some conclusions can be made:
\begin{itemize}
\item Voiced
	\begin{itemize}
	\item Quasiperiodic
	\item Dominant low frequency content
	\end{itemize}
\item Unvoiced
	\begin{itemize}
	\item Random 
	\item Dominant high frequency content
	\end{itemize}
\end{itemize}


There is not always a clearly distinction between unvoiced and voiced, e.g. during transitions there will be randomness from voiced to unvoiced, vice versa and quasiperiodicity, and it can therefore be difficult to judge if a signal frame is strictly voiced or unvoiced. 

Signal processing on speech is normally done on a frame by frame basis. The lenght of the frame is typically choosen such that the characteristics of the speech signal remain almost within a frame. A typical length of a frame would be between 20 - 30 ms where it can be stated that a speech signal will be Wide Sense Stationary (WSS).  

The human speech production system can be simple modelled using a white noise source and a time varying autoregressive (AR) filter as seen on \autoref{fig:modelSpeech}. 

\begin{figure}[H]
	\centering
	\includegraphics[width=0.6\textwidth]{figures/Speech/modelSpeech}
	\caption{Correspondence between the human speech production system with a simplified system based on time-varying filter.}
	\label{fig:modelSpeech}
\end{figure}      

By the use of linear prediction it is possible to estimate the coefficients of the time varying filter from an observed speech signal. 