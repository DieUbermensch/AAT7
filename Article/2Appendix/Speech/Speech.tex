\section{Speech} \label{sec:Speech}
This section will give a short introduction to speech and will show an examination of the characteristics of speech.

Speech is generated when an acoustical wave is radiated from the nostrils and the mouth. The acoustical wave is created as the air flow expelled from the lungs is pertubed by constrictions inside the body. 

Speech can be interpreted as an acoustic filter, with time varying frequency response. It is formed by the different parts of the human speech production system. Speech can be subclassified into two main classes; \textit{voiced} and \textit{unvoiced}. Voiced sounds are characterized by strong periodicity in the soundwave, here the fundamental frequency referred  to as the "pitch frequency", or simply as "pitch". The pitch ranges from 50 Hz to 250 Hz for men and 120 Hz to 500 Hz for women. Unvoiced sounds on the other hand are almost completely random. For a more detailed explanation refer to \citep{Speech}. 

An example of the word "problems" spoken by a man is given in \autoref{fig:voicedUnvoiced}, where the characteristics of voiced and unvoiced can be seen. 

\begin{figure}[H]
	\centering
	\includegraphics[width=0.6\textwidth]{figures/Speech/VoicedvsUnvoiced}
	\caption{Example of speech waveform of a male subject uttering the word ‘‘problems.’’ The expanded views of a voiced frame and an unvoiced frame are shown, with the magnitude of the Fourier Transform (FT) plotted. The frame is 256 samples in length.}
	\label{fig:voicedUnvoiced}
\end{figure}   

\newpage

From the example some conclusions can be made:
\begin{itemize}
\item Voiced
	\begin{itemize}
	\item Quasiperiodic
	\item Dominant low frequency content
	\end{itemize}
\item Unvoiced
	\begin{itemize}
	\item Random 
	\item Dominant high frequency content
	\end{itemize}
\end{itemize}


There is not always a clear distinction between unvoiced and voiced speech. An example would be during transitions between the two, there will be randomness from voiced to unvoiced and vice versa, along with quasiperiodicity. It can therefore be difficult to judge if a signal frame is strictly voiced or unvoiced. 

Signal processing on speech is normally done on a frame by frame basis. The length of the frame is typically chosen such that the characteristics of the speech signal remain almost within a single frame. A typical length of a frame would be 20 $m$s - 30 $m$s where it can be stated that a speech signal will be Wide Sense Stationary (WSS).  

The human speech production system can be modeled simply using a white noise source and a time varying autoregressive (AR) filter as seen on \autoref{fig:modelSpeech}. 

\begin{figure}[H]
	\centering
	\includegraphics[width=0.6\textwidth]{figures/Speech/modelSpeech}
	\caption{Correspondence between the human speech production system and a simplified system based on time-varying filter.}
	\label{fig:modelSpeech}
\end{figure}      

By the use of linear prediction (LP) it is possible to estimate the coefficients of the time varying filter from an observed speech signal 
\citep{Speech}.


