\section{Speaker Vibration and Frequency Response}\label{app:journal_1}

The purpose of this test is to examine if there is a correlation between the amount of vibration of a speaker and an increased amount of harmonic distortion. The purpose is also to examined if it is possible to detect when the coil hits the backplate. 

\subsection{Setup}

The setup of this experiment are depicted in Figure \ref{figure:SpeakertestSetup}, where the equipment is catalogued in \autoref{tab:UsedEquipment1}, and described as follows:

\begin{itemize}
\item Distortion and \gls{SPL} will be measured by a microphone at a distance of 1 meter in accordance with IEC 60268-5 Sound System Equipment - Part 5: Loudspeaker
\item Vibration will be measured by a Brüel \& Kjear Type 4344 accelerometer, placed at:
\begin{itemize}
\item The backplate of the lowest woofer
\item High, inside and on the back of the enclosure 
\end{itemize}
\item The speaker will be driven by a Crown Studio Reference I amplifier.
\item The ADC/DAC will convert measurements from accelerometers and microphone and relay to a computer via SPDIF.
\begin{itemize}
\item Both accelerometer and microphone is calibrated into outputting -37 dB at respectively 1 G and 94 dB \gls{SPL}.
\item All recordings are synchronised and timestamped with by looping the test file back into the converter.
\end{itemize}
\item The computer will be logging data with a RME HammerFall DIGI 96-PDST sound card and Adobe Audition.
\end{itemize}

Furthermore the speaker will be placed in the anechoic room to eliminate any external disturbances corresponding to the requirements demanded in the 
IEC 60268-5 standard.

\subsection*{Test Setup}

\subsection*{Equipment used and AAU-no.}

\begin{table}[H]
\centering
\ra{1.3}
\begin{tabular}{S[table-format=1]ccc} \toprule
    {Item} & {Description} & {AAU-no} \\ \bottomrule 
    1      &  B \& K Accelerometer Type 4319  & 06598   \\ 
    2      &  B \& K Accelerometer Type 4333  & 06596   \\ 
    3      &  B \& K 2-channel Accelerometer Pre-amp Type 2622  & 07013   \\
    4      &  B \& K Microphone Type 4165  & 08132   \\
    5      &  Gras - 26AK Pre-amp & 52665   \\
    6      &  B \& K Microphone Power supply Type 2804  & 07304   \\
    7      &  Crown Studio Reference I Amplifier & 52614   \\
    8      &  BEHRINGER digital A/D \& D/A Converter - Model ADA8000   & 56545   \\
    9      &  B \& K Accelerometer calibrator 4294 & 08023   \\
    10     &  B \& K Microphone calibrator 4231 & 78301   \\
    11     &  RME HammerFall DIGI 96-PDST sound card & 60919  \\
    11     &  Passive Dali Zensor 5 AX & NaN  \\ \bottomrule 
\end{tabular}
\caption{Table over equipment used in the test}
\label{tab:UsedEquipment1}
\end{table}
\vspace{-5mm}


\subsection{Procedure}\label{sec:SpeakerTestProcedure1}

The producer for this experiment is described as follows:
\vspace{-5mm}
\begin{enumerate}
\item Adjust volume on the amplifier to +3 dB gain.
\item Vacate the anechoic room and seal the room.
\item Start recording in Adobe Audition for:
\begin{itemize}
\item Driver and enclosure accelerometer
\item Microphone and playback loop
\end{itemize}
\item Playback file \path{chirp.wav}, which is a sinusoidal sweep from 2.4 kHz to 10 Hz. The duration of the sweep is 30 seconds and the sweep is linear. The amplitude is 0 dBFS.
\item After playback, stop recording all channels
\item Save recordings as \path{.wav} file
\item Enter room and adjust amplifier by +1 dB
\item Repeat step 2 through 7 until the speaker breaks.
\end{enumerate}

The \path{chirp.wav} are located at:\\
\scalebox{0.7}{
\path{CD://Maalinger/Maalinger030316 - Loudspeaker test/Measurements030316/chirp.wav}}\\




\subsection{Data Extraction}


It was possible to increase the volume by 19 dB until the loudspeaker unit got to damaged. The gain was linear increased by 1 dB, starting at 3 dB, until the 20th test were the loudspeaker unit failed. This results in 20 usable data sets.

The recordings can be found on:\\
\scalebox{0.7}{
\path{CD://Maalinger/Maalinger030316 - Loudspeaker test/Measurements030316/}}
And is indexed in folders \scalebox{0.8}{\path{Measure_X}}, where X corresponds to the test number. Every measurements inside is denoted as:
\begin{itemize}
\item Accelerometer on driver: \scalebox{0.8}{\path{Acc driver_0XX}}
\item Accelerometer in enclosure: \scalebox{0.8}{\path{Acc enclosure_0XX}}
\item Microphone: \scalebox{0.8}{\path{Mic_0XX}}
\item Playback loop: \scalebox{0.8}{\path{Reference_0XX}}
\end{itemize}

The Script used to create all graphs are located at:\\
\scalebox{0.7}{
\path{CD://Maalinger/Maalinger030316 - Loudspeaker test/Measurements030316/MeasurementAnalysis.m}}\\
Figures used in the following \ref{subsec:Raw_data_1} and Dataset 1 though 20 can be found compiled in a combined file at:
\scalebox{0.7}{
\path{CD://Maalinger/Maalinger030316 - Loudspeaker test/Measurements030316/AlleSamlet.pdf}}


\subsubsection{Raw Data}\label{subsec:Raw_data_1}

For each of the test four measurements was taken. These are the accelerometer placed on the driver and enclosure, the microphone and a reference to synchronize the measurements. 

As previously stated there are in total 20 datasets each containing four measurements, namely the vibrations from the enclosure and driver, the sound pressure from the microphone, and lastly a reference signal to synchronize all data. The amount of data is therefore large and only relevant datasets will be presented. The first dataset from the test is shown in \autoref{fig:raw1}.


%\begin{figure}[H]
%\centering
%\begin{subfigure}[t]{0.335\textwidth}
%	\tikzsetnextfilename{Raw_Driver1}
%	\input{figures/Forsog1/Driver1.tex}
%	\caption{Vibration from driver.}
%	\label{fig:raw_driver1}
%\end{subfigure}
%\begin{subfigure}[t]{0.3\textwidth}
%	\tikzsetnextfilename{Raw_Enclosure1}
%	\input{figures/Forsog1/Enclosure1.tex}
%	\caption{Vibration from enclosure.}
%	\label{fig:raw_enclosure1}
%\end{subfigure}
%\begin{subfigure}[t]{0.3\textwidth}
%	\tikzsetnextfilename{Raw_microphone1}
%	\input{figures/Forsog1/Microphone1.tex}
%	\caption{Sound pressure from microphone.}
%	\label{fig:raw_microphone1}
%\end{subfigure}
%\caption{The measured data of (a) the vibration on the driver, (b) the vibration on the enclosure, and (c) the sound pressure from the microphone. Dataset 1.}
%\label{fig:raw1}
%\end{figure} 


Looking at the first dataset reveals that there are mechanical issues with the loudspeaker. In \autoref{fig:raw_driver1} there is a strong indication of a resonance frequency at 25 seconds with a peak amplitude on approximately 0.02. A similar observation can be found in \autoref{fig:raw_enclosure1} which shows the vibration measured by the accelerometer placed on the enclosure where a resonance frequency appears after 25 seconds. Another mechanical resonance frequency can be found located after 14 seconds for the enclosure. The sound pressure level measured by the microphone is in contrast fairly linear compared to the measured vibrations. Dataset 10 is shown in \autoref{fig:raw10} where the gain is increased by 10 dB.

%\begin{figure}[H]
%\centering
%\begin{subfigure}[t]{0.335\textwidth}
%	\tikzsetnextfilename{Raw_Driver10}
%	\input{figures/Forsog1/Driver10.tex}
%	\caption{Vibration from driver.}
%	\label{fig:raw_driver10}
%\end{subfigure}
%\begin{subfigure}[t]{0.3\textwidth}
%	\tikzsetnextfilename{Raw_Enclosure10}
%	\input{figures/Forsog1/Enclosure10.tex}
%	\caption{Vibration from enclosure.}
%	\label{fig:raw_enclosure10}
%\end{subfigure}
%\begin{subfigure}[t]{0.3\textwidth}
%	\tikzsetnextfilename{Raw_microphone10}
%	\input{figures/Forsog1/Microphone10.tex}
%	\caption{Sound pressure from microphone.}
%	\label{fig:raw_microphone10}
%\end{subfigure}
%\caption{The measured data of (a) the vibration on the driver, (b) the vibration on the enclosure, and (c) the sound pressure from the microphone. Dataset 10.}
%\label{fig:raw10}
%\end{figure} 

The measurements from dataset 10 is very similar to the measurements seen in dataset 1. Since no particular change is happening between dataset 1 to 10 it is assumed that the performance of the loudspeaker remains good. A further analysis of the performance speaker will be done in the analysis of the harmonic distortion section to examine the amount of harmonic distortion present. Dataset 14 is shown in \autoref{fig:raw14}. 

%\begin{figure}[H]
%\centering
%\begin{subfigure}[t]{0.335\textwidth}
%	\tikzsetnextfilename{Raw_Driver14}
%	\input{figures/Forsog1/Driver14.tex}
%	\caption{Vibration from driver.}
%	\label{fig:raw_driver14}
%\end{subfigure}
%\begin{subfigure}[t]{0.3\textwidth}
%	\tikzsetnextfilename{Raw_Enclosure14}
%	\input{figures/Forsog1/Enclosure14.tex}
%	\caption{Vibration from enclosure.}
%	\label{fig:raw_enclosure14}
%\end{subfigure}
%\begin{subfigure}[t]{0.3\textwidth}
%	\tikzsetnextfilename{Raw_microphone14}
%	\input{figures/Forsog1/Microphone14.tex}
%	\caption{Sound pressure from microphone.}
%	\label{fig:raw_microphone14}
%\end{subfigure}
%\caption{The measured data of (a) the vibration on the driver, (b) the vibration on the enclosure, and (c) the sound pressure from the microphone. Dataset 14.}
%\label{fig:raw14}
%\end{figure} 

While most of the measurements show the exact same characteristics as previous datasets, a slight change occurs in the end of the vibration measured on the driver in \autoref{fig:raw_driver14} and the sound pressure level in \autoref{fig:raw_microphone14}. After approximately 29 seconds a peak is observed in \autoref{fig:raw_driver14} which could indicate that the coil at that point hit the back plate of the driver. A further examination will be made in a later section. The measurements from the microphone, also show that a peak occurs at the same point, which most likely will cause the amount of harmonic distortion to increase. Dataset 19, which is the last dataset before the loudspeaker break down, is shown in \autoref{fig:raw19}.

%\begin{figure}[H]
%\centering
%\begin{subfigure}[t]{0.335\textwidth}
%	\tikzsetnextfilename{Raw_Driver19}
%	\input{figures/Forsog1/Driver19.tex}
%	\caption{Vibration from driver.}
%	\label{fig:raw_driver19}
%\end{subfigure}
%\begin{subfigure}[t]{0.3\textwidth}
%	\tikzsetnextfilename{Raw_Enclosure19}
%	\input{figures/Forsog1/Enclosure19.tex}
%	\caption{Vibration from enclosure.}
%	\label{fig:raw_enclosure19}
%\end{subfigure}
%\begin{subfigure}[t]{0.3\textwidth}
%	\tikzsetnextfilename{Raw_microphone19}
%	\input{figures/Forsog1/Microphone19.tex}
%	\caption{Sound pressure from microphone.}
%	\label{fig:raw_microphone19}
%\end{subfigure}
%\caption{The measured data of (a) the vibration on the driver, (b) the vibration on the enclosure, and (c) the sound pressure from the microphone. Dataset 19.}
%\label{fig:raw19}
%\end{figure} 

Dataset 19 has the same characteristics as dataset 14, though with a more noticeable peak at 29 seconds for both the measurements from the driver and the microphone. In the last dataset shown in \autoref{fig:raw20}, the loudspeaker breaks down.


%\begin{figure}[H]
%\centering
%\begin{subfigure}[t]{0.335\textwidth}
%	\tikzsetnextfilename{Raw_Driver20}
%	\input{figures/Forsog1/Driver20.tex}
%	\caption{Vibration from driver.}
%	\label{fig:raw_driver20}
%\end{subfigure}
%\begin{subfigure}[t]{0.3\textwidth}
%	\tikzsetnextfilename{Raw_Enclosure20}
%	\input{figures/Forsog1/Enclosure20.tex}
%	\caption{Vibration from enclosure.}
%	\label{fig:raw_enclosure20}
%\end{subfigure}
%\begin{subfigure}[t]{0.3\textwidth}
%	\tikzsetnextfilename{Raw_microphone20}
%	\input{figures/Forsog1/Microphone20.tex}
%	\caption{Sound pressure from microphone.}
%	\label{fig:raw_microphone20}
%\end{subfigure}
%\caption{The measured data of (a) the vibration on the driver, (b) the vibration on the enclosure, and (c) the sound pressure from the microphone. Dataset 20.}
%\label{fig:raw20}
%\end{figure} 

As seen in \autoref{fig:raw_driver20} the loudspeaker unit breaks down at the mechanical resonance frequency for the driver. The reason to why vibrations on the enclosure and sound pressure are still presentthe  after 25 seconds is because the other loudspeaker unit did not entirely break down. Since the loudspeaker unit broke down at the mechanical driver resonance frequency, it would be relevant to examine if there is any correlation as such.



\subsection{Analysis}

\subsubsection{Frequency response}

To determine the frequency response of the vibration from the driver, enclosure and the speaker frequency response a Fast Fourier Transformation (FFT) is applied to all three measurements.

%\begin{figure}[H]
%\centering
%\begin{subfigure}[t]{0.37\textwidth}
%	\tikzsetnextfilename{FFT_driver1}
%	\input{figures/FFT_driver1.tex}
%	\caption{Driver.}
%	\label{fig:FFT_driver1}
%\end{subfigure}
%\begin{subfigure}[t]{0.28\textwidth}
%	\tikzsetnextfilename{FFT_enclosure1}
%	\input{figures/FFT_enclosure1.tex}
%	\caption{Enclosure.}
%	\label{fig:FFT_enclosure1}
%\end{subfigure}
%\begin{subfigure}[t]{0.32\textwidth}
%	\tikzsetnextfilename{FFT_mic1}
%	\input{figures/FFT_mic1.tex}
%	\caption{Microphone.}
%	\label{fig:FFT_mic1}
%\end{subfigure}
%\caption{Frequency response of (a) the vibration on the driver, (b) the vibration on the enclosure, and (c) the speaker. Test 1.}
%\label{fig:FFT1}
%\end{figure}

From \autoref{fig:FFT_driver1} it is seen that the driver has a resonance frequency located at approximately 390 Hz and a dip at 1290 Hz. The enclosure also has a peak located at 390 Hz, but also has another peak located at 1290 Hz which is the opposite to the frequency response of the driver. The speaker frequency response measured by the microphone, shows that the response is flat compared to the two other responses from 10 Hz to 2.4 kHz.

%\begin{figure}[H]
%\centering
%\begin{subfigure}[t]{0.37\textwidth}
%	\tikzsetnextfilename{FFT_driver10}
%	\input{figures/FFT_driver10.tex}
%	\caption{Driver.}
%	\label{fig:FFT_driver10}
%\end{subfigure}
%\begin{subfigure}[t]{0.28\textwidth}
%	\tikzsetnextfilename{FFT_enclosure10}
%	\input{figures/FFT_enclosure10.tex}
%	\caption{Enclosure.}
%	\label{fig:FFT_enclosure10}
%\end{subfigure}
%\begin{subfigure}[t]{0.32\textwidth}
%	\tikzsetnextfilename{FFT_mic10}
%	\input{figures/FFT_mic10.tex}
%	\caption{Microphone.}
%	\label{fig:FFT_mic10}
%\end{subfigure}
%\caption{Frequency response of (a) the vibration on the driver, (b) the vibration on the enclosure, and (c) the speaker. Test 10.}
%\label{fig:FFT10}
%\end{figure}

The frequency response for all three measurements in test 10 is very similar to the frequency response found in test 1. The same applies to all the test from 1 to 10. No noticeable difference between the test is observed.

%\begin{figure}[H]
%\centering
%\begin{subfigure}[t]{0.37\textwidth}
%	\tikzsetnextfilename{FFT_driver19}
%	\input{figures/FFT_driver19.tex}
%	\caption{Driver.}
%	\label{fig:FFT_driver19}
%\end{subfigure}
%\begin{subfigure}[t]{0.28\textwidth}
%	\tikzsetnextfilename{FFT_enclosure19}
%	\input{figures/FFT_enclosure19.tex}
%	\caption{Enclosure.}
%	\label{fig:FFT_enclosure19}
%\end{subfigure}
%\begin{subfigure}[t]{0.32\textwidth}
%	\tikzsetnextfilename{FFT_mic19}
%	\input{figures/FFT_mic19.tex}
%	\caption{Microphone.}
%	\label{fig:FFT_mic19}
%\end{subfigure}
%\caption{Frequency response of (a) the vibration on the driver, (b) the vibration on the enclosure, and (c) the speaker. Test 10.}
%\label{fig:FFT19}
%\end{figure} 

The frequency responses test 19, which is the test before the loudspeaker got damaged, shows little signs of any changes in the frequency responses from the previous tests, thus is not possible to determine if the loudspeaker is close the being damaged alone from the frequency responses.


\subsubsection{Harmonic distortion}

In this section the amount of harmonic distortion of the loudspeaker is analysed, to examine if it is possible to determine the performance of the loudspeaker from harmonics. To analyse the harmonic distortion a spectrogram is used. The spectrogram is a three dimensional plot which shows the frequency, time and magnitude. Basically the spectrogram is an array of multiple FFT each calculated at a given time with a given window.

The advantages of using a spectrogram is that it gives a good overview of the spectral content in the signal at any given time. Since the harmonic frequencies depends on the fundamental frequency, it is better to use a spectrogram rather than a regular FFT, since the spectrogram will reveal all harmonic distortion for all fundamental frequency from 10 Hz to 2.4 kHz. The spectrograms for microphone in dataset 1 and 19 are shown in \autoref{fig:spec_mic}.

%\begin{figure}[H]
%\centering
%\begin{subfigure}[t]{0.47\textwidth}
%	\tikzsetnextfilename{Microphone_spec1}
%	\input{figures/Forsog1/Microphone_spec1.tex}
%	\caption{Microphone dataset 1}
%	\label{fig:spectrogram_mic1}
%\end{subfigure}
%\begin{subfigure}[t]{0.47\textwidth}
%	\tikzsetnextfilename{Microphone_spec19}
%	\input{figures/Forsog1/Microphone_spec19.tex}
%	\caption{Microphone dataset 19}
%	\label{fig:spectrogram_mic19}
%\end{subfigure}
%\caption{The spectrograms of the microphone dataset 1 and 19. The prominent red line is the sine sweep from 2.4 kHz to 10 Hz while the yellow and blue lines along the red line are harmonic distortion.}
%\label{fig:spec_mic}
%\end{figure} 

The spectrograms of dataset 1 and 19 for the microphone, seen in \autoref{fig:spectrogram_mic1} and \autoref{fig:spectrogram_mic1}, show a prominent orange line and three weak blue lines above the red line. It is seen that the orange line is a linear decreasing function from 2.4 kHz to 10 Hz where the frequency is a function of time. This indicates that the red line is the fundamental frequency of the sine sweep from 2.4 kHz to 10 Hz. The yellow line are linear functions of the harmonic distortions. The spectrograms clearly show that increasing the gain will increase the amount of harmonic distortion as well. An interesting observation is the spectral frequency leak that occurs at lower frequencies revealing large amount of distortion at low frequency and high gain. Since the power of the spectrum other than the harmonic frequencies increase, it could indicate that the coil hit the back plate of the driver at this point. From the analysis it can be concluded that the large amount of distortion can indicate that the coil is hitting the back plate of the driver.

The next part will be to analyse if similar observation can be found in the spectrogram for vibration measurements on the driver and the enclosure. 

%\begin{figure}[H]
%\centering
%\begin{subfigure}[t]{0.47\textwidth}
%	\tikzsetnextfilename{Driver_spec1}
%	\input{figures/Forsog1/Driver_spec1.tex}
%	\caption{Driver dataset 1.}
%	\label{fig:spectrogram_driver1}
%\end{subfigure}
%\begin{subfigure}[t]{0.47\textwidth}
%	\tikzsetnextfilename{Driver_spec19}
%	\input{figures/Forsog1/Driver_spec19.tex}
%	\caption{Driver dataset 19.}
%	\label{fig:spectrogram_driver19}
%\end{subfigure}
%\caption{The spectrograms of the driver dataset 1 and 19. The prominent orange line is the sine sweep from 2.4 kHz to 10 Hz while the yellow lines are harmonic distortion.}
%\label{fig:spec_driver}
%\end{figure} 

The spectrograms of dataset 1 and 19 of the driver are seen in \autoref{fig:spec_driver}. From the spectrograms it is seen that the harmonic distortion is also present in the driver measurements as well as the frequency leak at lower frequencies at high gain. Finally the spectrum of the enclosure measurements is examined in \autoref{fig:spec_driver}.

%\begin{figure}[H]
%\centering
%\begin{subfigure}[t]{0.47\textwidth}
%	\tikzsetnextfilename{Enclosure_spec1}
%	\input{figures/Forsog1/Enclosure_spec1.tex}
%	\caption{Enclosure dataset 1.}
%	\label{fig:spectrogram_enclosure1}
%\end{subfigure}
%\begin{subfigure}[t]{0.47\textwidth}
%	\tikzsetnextfilename{Enclosure_spec19}
%	\input{figures/Forsog1/Enclosure_spec19.tex}
%	\caption{Enclosure dataset 19}
%	\label{fig:spectrogram_enclosure19}
%\end{subfigure}
%\caption{The spectrograms of the enclosure dataset 1 and 19. The prominent orange line is the sine sweep from 2.4 kHz to 10 Hz while the yellow lines are harmonic distortion.}
%\label{fig:spec_enclosure}
%\end{figure} 

It is seen that there is considerately more frequency leak throughout the whole spectrum compared to the spectrogram for the driver and microphone. However the measurements from the driver, enclosure, and microphone is the large amount of frequency leak at very low frequencies seen at after 29 seconds.

This section concludes that amount of distortion increases significantly when the gain is increased too. Also, it can be concluded that there is a frequency leak at lower frequencies on both driver, enclosure, and microphone. 



\subsection{Back Plate Hit Detection}\label{sec:hit_detect}

As previously stated, it is possible that the loudspeaker coil hits the back plate of the driver if the coil moves too far back. In the long run this will damage the loudspeaker, and should therefore be avoided. In this section an analysis on how to detect these hits, will be provided. 

The frequency response of a mechanical frame of a loudspeaker has the same characteristics as a low-pass filter. This means, if the coil is hitting the back plate of the driver, it will result in a increase in energy in the low frequency spectrum. As most music and loudspeakers have most of its energy above 50 Hz, it can be assumed that only a back plate hit will generate vibrations at very low frequencies between, for instance, 0 Hz to 20 Hz. To examine if this is true, a section of the dataset 19 of the driver, where a hit is suspected to have occurred, is analysed. The analysed section is seen in \autoref{fig:raw_driver19_windows}.


%\begin{figure}[H]
%\centering
%\begin{subfigure}[t]{0.55\textwidth}
%	\tikzsetnextfilename{raw_driver19_window}
%	\input{figures/raw_driver19_window.tex}
%	\caption{Raw dataset 19 for driver.}
%	\label{fig:raw_driver19_window}
%\end{subfigure}
%%\hspace{6mm} 
%\begin{subfigure}[t]{0.43\textwidth}
%	\tikzsetnextfilename{raw_driver19_window_zoom}
%	\input{figures/raw_driver19_window_zoom.tex}
%	\caption{The grey area zoomed in.}
%	\label{fig:raw_driver19_window_zoom}
%\end{subfigure}
%\caption{The greyed area is analysed to examine if there are any signs of an impulse.}
%\label{fig:raw_driver19_windows}
%\end{figure}

The same window is applied to all datasets for the driver and afterwards spectrum analysed to reveal any increase in low frequent energy. 

%\begin{figure}[H]
%\centering
%\tikzsetnextfilename{FFTFinal}
%\input{figures/Forsog1/FFTFinal.tex}
%\caption{The FFT of the measurements at 80 Hz.}
%\label{fig:FFT_hit}
%\end{figure}

It is seen that the energy for non-harmonic frequencies is increased at very low frequencies as indicated with the arrow. The yellow graph is the first dataset and the red is the tenth dataset. As no hit was observed in the window at dataset 1 and 10, the energy of lower part of the spectrum was not increased. However the suspected hit in dataset 19 has an energy increase in the low frequency spectrum. Note that since the window is 0.25 seconds wide, frequencies below 4 Hz should be discarded.

\subsection{Error sources}

Few errors could have occurred during trials, since the only variable adjusted during between measurements was the gain of the amplifier. There were however problems with moving the setup from the control room were it was calibrated and then moved into the anechoic room. The calibration was done using an visual equalizer where the output was shown with a bar chart hence the data should bee seen with a tolerance of +/- 0.5 dB.

\subsection{Conclusion}
The test concludes that is possible to detect a hit against the back plate. The draw back of using an accelerometer is however that it is not possible to determine a hit before it happens. 
