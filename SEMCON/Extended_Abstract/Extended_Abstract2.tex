% This file was converted to LaTeX by Writer2LaTeX ver. 1.2
% see http://writer2latex.sourceforge.net for more info
\documentclass[a4paper]{article}
\usepackage[utf8]{inputenc}
\usepackage[T1]{fontenc}
\usepackage[english, danish]{babel}
\usepackage{amsmath}
\usepackage{amssymb,amsfonts,textcomp}
\usepackage{color}
\usepackage[top=2.54cm,bottom=2.54cm,left=2.54cm,right=2.54cm,nohead,nofoot]{geometry}
\usepackage{array}
\usepackage{caption}
\usepackage{hhline}
\usepackage{hyperref}
\hypersetup{colorlinks=true, linkcolor=blue, citecolor=blue, filecolor=blue, urlcolor=blue}
% Footnote rule
%\setlength{\skip\footins}{0.119cm}
%\renewcommand\footnoterule{\vspace*{-0.018cm}\setlength\leftskip{0pt}\setlength\rightskip{0pt plus 1fil}\noindent\textcolor{black}{\rule{0.25\columnwidth}{0.018cm}}\vspace*{0.101cm}}
\title{}
\begin{document}
	%\[%\clearpage\clearpage\setcounter{page}{1}{\centering\color{black}
	
	{\centering
		\subsection*{Active Noise Control of Speech in Headphones using Linear Prediction}}
	
	{\centering
		\textit{Christian Claumarch, Kasper Kiis Jensen$^{\star \star}$, Oliver Palmhøj Jokumsen$^{\star}$, \\Maxime Démurger, Mikkel Krogh Simonsen}
		\par}
	{\centering
		\textbf{Acoustics and Audio Technology Group 761}
		\par}
	
	\bigskip
	
	\paragraph{Introduction} 
	Active Noise Control (ANC) is a widely used technique in consumer headphones for attenuating noise. ANC is useful for attenuation of periodic noise e.g. machinery but has limited ability to attenuate quasiperioc noise e.g. speech (50 Hz -- 4000 Hz). This paper therefore focuses on improving attenuation of speech. Feedforward ANC systems are widely used, where an FIR-filter is adapted by a Filtered-$x$ Least Mean Square (FXLMS)-algorithm. The main problem when implementing ANC feedforward systems is delays in converters. A tested $\Sigma\Delta$-converter has a delay of 225 $\mu$s -- 900 $\mu$s. Therefore a Linear Prediction (LP) method combined with multirate processing is proposed to compensate for the introduced conversion delays.

	\paragraph{Methods and Proposals}
	The feedforward FXLMS algorithm uses a control filter which coefficients are adapted by a FXLMS optimization algorithm to output a counterphase signal.  

	For LP Wiener filtering is used in cascade to predict 10 samples which is outputted to the ANC system.  

	\paragraph{Simulation Results}
	The combined LP FXLMS system 

	\paragraph{Discussion}
	 No real time implementation is attempted because the computation cost of the LP is >50,000 instructions per sample.
	
	
	\paragraph{References}
	\begin{description}
		\item{[1]}		{SpeechCoding} Wai C. Chu: \emph{Speech Coding Algorithms}, Wiley $1^{st}$ Edition 2013
		\item{[2]}		{HansenSnyder} Colin Hansen and Scott Snyder and Xiaojun Qiu and Laura Brooks and Danielle Moreau  : \emph{Active Control of Noise and Vibration}, CRC Press $2^{nd}$ Edition 2012
	\end{description}
\end{document}