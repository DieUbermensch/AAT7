\documentclass[12pt,a4paper,openright]{article}
\usepackage[utf8]{inputenc}
\usepackage{amsmath}
\usepackage{amsfonts}
\usepackage{amssymb}
\usepackage{graphicx}
\usepackage[left=2.00cm, right=2.00cm, top=2.00cm, bottom=2.00cm]{geometry}
\usepackage{float}
\usepackage{mathtools,xparse}
\newcommand{\Lagr}{\mathcal{L}}
\DeclarePairedDelimiter{\abs}{\lvert}{\rvert}
\DeclarePairedDelimiter{\norm}{\lVert}{\rVert}
\NewDocumentCommand{\normL}{ s O{} m }{%
	\IfBooleanTF{#1}{\norm*{#3}}{\norm[#2]{#3}}_{L_2(\Omega)}%
}
\begin{document}
\title{Poster Presentation}

Hey guys, we're group 761 of Acoustics and Audio Technology, and it's my pleasure to show off what we've been doing for the past months.

\section{Introduction}	
Out project is about Active Noise Control, for those of you who know of ANC headsets, you'll probably know that they're not good at canceling out peoples voices.
 - In fact it seem to enhance them.\\
And you all probably know the very basic idea of how noise canceling works, where you measure a incoming unwanted sound, produce the phase-inversed copy of it, and play it through a speaker, in order to cancel the original unwanted signal.
Generally people seem to use feedforward systems for this, and even though this topology is capable of canceling speech, it needs some help. Because when you do a digital implementation you get some delays in the system. But we're able to cope with this delay, using Linear Prediction. - This is just a teaser for now.

\section{Methods}	
The method for doing ANC is with an adaptive filter - a filter whose coefficients are altered with an algorithm - the FXLMS.
Now back to the LP - to help with the delay problem we're introducing a LP scheme:
Figure 1 describes that the wiener filter is used for predicting future samples, notice the symbols describing the Wiener-Hopf equation is adapting the Wiener-filter.\\
$\Hat{R}\bar{a} = -\bar{\hat{r}}_x$ \\
As those familiar with matrix computation will know - doing the inverse of a matrix takes a lot of iterations, and is as such very heavy computationally wise, and as such we're using the Levinson-Durbin method to estimate the inverse of the $C_{xx}$ - matrix, which is then input in the Wiener-hopf and then along it goes in the system.
And here we can see where we actually do the ANC - with the control-filter adapted, and with the TF of the cancellation path, which is the path from 2 to 3.
	
\section{Results of Simulation}
The first simulation was a "brute force" test, where we tested the Prediction Gain, PG) for various different Framelengths, N and Overlap, O. Where a higher PG obviously is better, i.e. where the graph is most yellow.
We found the optimal parameters to be $N=1200$ and $O=1100$.\\\\
And we made a comparison of the FXLMS vs. the LP FXLMS. As we can see, with no delays the FXLMS is capable of achieving an infinite attenuation, so in a perfect world we wouldn't need LP - however this world is obviously not perfect, so we \textit{need} the LP. The red line illustrates the performance of that one.\\\\
Lastly we've compared the frequency responses to each other, to determine whether the LP altered the characteristics of the signal, not how there is up to 30 dB in improves attenuation for the LP.\\

\section{Discussion}
Something that might be a drawback is the computational cost of this scheme, it is expensive to run LP with FXLMS, in fact it costs 15000 instructions pr. sample. A way of working around this problem, is to utilize multirate.\\
It should be noted that for some high frequencies, for FXLMS, the delay wraps around and at some point our counterphase signal become counter-counter-phase and actually gains the noise. Another reason to get rid of delays.

\section{Conclusion}
All in all the proposed tech works, and it was possible to attenuate speech greatly - up to 30 dB even, for delay of 10 samples.\\
However it proved to be heavy on computational cost, and was therefor not feasible to implement on a DSP - though got it to work in simulation, however not in real-time.
But if you would like to step over to this rack, you can scan the QR-code, put on a pair of headphones and then you can listen for yourself. It's pretty cool if I have to say it myself.
\end{document}
